
\documentclass [12pt , a4paper] {report}

\usepackage{graphicx}      % include this line if your document contains figures
\usepackage{amsmath}%
\usepackage{amsfonts}%
\usepackage{amssymb}%
\usepackage{color}
\usepackage{psfrag}
\usepackage[autolinebreaks,useliterate]{mcode}
\usepackage{multirow}

%===============================================================================
\begin{document}

%\lstset{language= Matlab , basicstyle=\small \ttfamily, stringstyle=\color{magenta},  commentstyle=\color{green}}

%\setlength{\parindent}{0pt}
%\setlength{\parskip}{1ex plus .5ex minus 0.2ex}
%\linespread{1.5}

\begin{titlepage}

\vspace{1cm}

\noindent{\bf \Huge  Frequency-Domain \\ Robust Controller Design\\}

\vspace*{0.5cm}
\noindent{\bf \LARGE  A Toolbox for MATLAB\\}

\vspace{6cm}

\noindent{\bf \Large  User Manual\\}

\noindent{\large Version 1.0}

\vspace{2cm}


\begin{center}

\includegraphics[width = 3.5cm]{LA_red.eps}\hspace{5cm}\includegraphics[width = 3.5cm]{Logo_epfl.eps}

\vspace*{0.5cm}

{Automatic Control Laboratory\\
EPFL, Switzerland\\
May 2012}
\end{center}
\end{titlepage}

%===============================================================================

\tableofcontents

\chapter{Introduction}
<<<<<<< HEAD
The Frequency-Domain Robust Controller Design Toolbox is a tool for designing robust linearly parameterized controllers and SISO rational controllers in the Nyquist diagram. It can be used to design controllers of any order for parametric models or nonparametric models obtained for example by the identification toolbox of MATLAB. The robust controllers are designed in terms of $H_\infty$ performance or classical robustness margins such as the gain and phase margin, for single/multi-model, SISO/MIMO systems.  The toolbox also supports designing gain-scheduled controllers. 
=======
The Frequency-Domain Robust Controller Design Toolbox is a tool for designing robust linearly parameterized controllers in the Nyquist diagram. It can be used to design linearly parameterized controllers of any order for parametric models or nonparametric models obtained for example by the identification toolbox of MATLAB. The robust controllers are designed in terms of $H_\infty$ performance or classical robustness margins such as the gain and phase margin, for single/multi-model, SISO/MIMO systems.  The toolbox also supports designing gain-scheduled controllers. 
>>>>>>> statespace

In all of design cases, linear or convex optimization problems are solved. For linear and quadratic optimization the well-known \texttt{linprog} or \texttt{quadprog} (depending on the problem) commands of the Optimization toolbox of MATLAB are used. While convex optimization problems are formulated with YALMIP and can be solved with all available solvers (e.g. \texttt{sdpt3} is used as a default solver) . Many commands of the Control toolbox of MATLAB are used as well.

In this manual as well as describing theoretical bases of the optimization problems in a terse manner, our main attempt is to provide a quite comprehensive set of examples to exhibit wide functionality, yet user-friendliness of this tool.  The theoretical bases used in this toolbox have been published in  \cite{KKL07, KKL07e,KG10,GKL10b}. The development of the toolbox was started by the works of Vinicius de Oliveira and continued by Mehdi Sadeghpoor during an internship in Automatic Control Laboratory \cite{SOK12}.

In this chapter, the theoretical bases of the design methods used in the toolbox are presented. The description of the optimization problems that are solved in the toolbox for different desired control performance  (GPhC, loop shaping, and $H_{\infty}$), is the main subject of this section. The extension of these methods to MIMO systems and gain-scheduled controller design, is described. In the sequel, first the class of controllers and models are presented and then different control performances that can be considered by the toolbox are given.

\section{Class of models and controllers}

\subsection{Models}

Models can be parametric or nonparametric. The design method, in fact, needs the frequency response of the plant model in a finite number of frequencies which can be obtained directly from data by spectral analysis (for example by the identification toolbox of Matlab) or computed from a parametric model. Therefore, high order models with pure time delay and non minimum phase zeros can be considered with no approximation. Thus, we define the set of models:  $\textsl{M}=\{G_i(j\omega), \: i=1,\ldots,m \}$ where $G_i(j\omega)$ can be a scalar or a matrix function representing the frequency response of a SISO or MIMO model, respectively, over a vector of frequency points $\omega_i=[\omega_{i_1}, \omega_{i_2}, \ldots, \omega_{i_{N_i}}]$. $N_i$ is large enough to give a good approximation of the frequency response of the system $G_i$. The methods are described on a SISO system. It will also be explained how these methods can be applied in designing multivariable decoupling controllers for MIMO systems.  

\subsection{Controllers}

<<<<<<< HEAD
The toolbox designs linearly parameterized controllers and SISO rational controllers.

A linearly parameterized controller has the following form:
=======
The toolbox designs linearly parameterized controllers. A linearly parameterized controller has the following form:
>>>>>>> statespace
\begin{equation}
\label{linparcon}
\rho^T \phi(s)=[\rho_1 \: \: \rho_2 \: \: \ldots \:\: \rho_n]\times [\phi_1(s) \: \: \phi_2(s) \:\: \ldots \:\: \phi_n(s)]^T
\end{equation}
<<<<<<< HEAD
where $\rho_1$, $\rho_2$, $\ldots$, $\rho_n$ are controller parameters and $\phi_1(s)$, $\phi_2(s)$, $\ldots$, $\phi_n(s)$ are basis transfer functions. These transfer functions must be stable, i.e. with no right half plane poles.

For SISO systems, a rational controller with the following form can be defined:

\begin{equation}
    K(j\omega) = \frac{X(j\omega)}{Y(j\omega)} = \frac{\rho_x^T \phi (s)}{f(s) \ \rho_y^T \phi (s)}
\end{equation}

$\phi$ is a vector of Laguerre or generalized basis functions, described below, and $f(s)$ is a fixed part of the controller. $\rho_x$ and $\rho_y$ are vectors of controller parameters.
=======
where $\rho_1$, $\rho_2$, $\ldots$, $\rho_n$ are controller parameters and $\phi_1(s)$, $\phi_2(s)$, $\ldots$, $\phi_n(s)$ are basis transfer functions. These transfer functions must be stable, i.e. with no right half plane poles. 
>>>>>>> statespace

\subsubsection{PID, PI, PD:}

The proportional-integral-derivative (PID) controller is a familiar case of these types of controllers with 3 parameters $[\rho_1 \:\, \rho_2 \:\, \rho_3]=[k_p \:\, k_i \:\, k_d]$ and the vector of basis transfer functions:
\begin{equation}
\phi (s)=[1 \:\: \frac{1}{s} \:\: \frac{s}{1+\tau s}]^T
\end{equation}
for continuous-time systems and:
\begin{equation}
\phi (z)=[1 \:\: \frac{z}{z-1} \:\: \frac{z-1}{z}]^T 
\end{equation}
for discrete-time systems, where $\tau$ is the time constant of the derivative part in the continuous-time one. PI and PD are special cases of the above equations. Besides these common linearly parameterized controllers, this toolbox supports the following higher order types, too: 

\subsubsection{Laguerre basis functions:}

\begin{equation}
\phi_1(s)=1, \quad \phi_i(s)=\frac{\sqrt{2\xi}(s-\xi)^{i-2}}{(s+\xi)^{i-1}} \quad   i=2,3,\ldots,n+1 
\end{equation}
for continuous-systems and:
\begin{equation}
\phi_1(z)=1, \quad \phi_i(z)=\frac{\sqrt{1-a^2}}{z-a} \left(\frac{1-az}{z-a}\right)^{i-2} \quad   i=2,3,\ldots,n+1 
\end{equation}
for discrete-time systems, where $n$ is the order of the controller, $\xi>0$, and $-1<a<1$.

\subsubsection{Generalized orthonormal basis functions:}

\begin{equation}
\phi_1(s)=1, \quad \phi_i(s)=\frac{\sqrt{2R_e(\xi_{i-1})}}{s+\xi_{i-1}} \prod_{k=1}^{i-2}\frac{s-\bar{\xi_k}}{s+\xi_k} \quad  i=2,3,\ldots,n+1 
\end{equation}
for continuous-time systems and
\begin{equation}
\phi_1(z)=1, \quad \phi_i(z)=\frac{\sqrt{1-|\xi_{i-1}|^2}}{z-\xi_{i-1}} \prod_{k=1}^{i-2}\frac{1-\bar{\xi_k}z}{z-\xi_k} \quad  i=2,3,\ldots,n+1 
\end{equation}
for discrete-time systems,
where $\xi_1, \ldots, \xi_n$ are complex numbers, $R_e$ denotes the real part of a complex number, and $\bar{\xi_k}$ is the complex conjugate of $\xi_k$.

\subsubsection{State space:}
A state space controller has the following form:

\begin{equation}
    K(s) = C \left(s I - A\right)^{-1} B + D
\end{equation}
for continuous-time systems and 

\begin{equation}
    K(z) = C \left(z I - A\right)^{-1} B + D
\end{equation}
for discrete-time systems.

In the toolbox, the eigenvalues of the A matrix (equivalent to the poles of the system) and the B or C matrix are specified. $\rho$ represents the entries of the B or C matrix (whichever was not specified) and of the D matrix.

<<<<<<< HEAD
A state space representation of any of the previous controller types (PID, Laguerre etc.) is possible by setting the eigenvalues of the A matrix accordingly.




=======
A state space representation of any of the previous controller types (PID, Laguerre etc.) is possible by setting the eigenvalues of the A matrix accordingly. 
>>>>>>> statespace

As stated earlier, one can define one's own vector of basis functions, $\phi$, of any desired order as well as the above listed types.

\begin{figure}
\centering
\psfrag{a}{$\beta$}
\psfrag{c}{$d_1$}
\psfrag{d}{$d_2$}
\psfrag{e}{$\omega_x$}
\psfrag{b}{$\omega_c$}
\psfrag{p}{$\varphi_m$}
\psfrag{q}{\small $-1/g_m$}
\includegraphics[width=0.65\linewidth]{figure2}
\caption{GPhC specifications converted to linear constraints in Nyquist diagram}
\label{fig:GPhC}
\end{figure}

\section{GPhC controller}
Gain margin, phase margin and crossover frequency (GPhC) are typical performance specifications for PID controller design in industry. We use these specifications for SISO stable systems if the number of integrators in the open-loop transfer function is less than or equal to 2. Specifying the gain and phase margin defines a straight line in the Nyquist diagram (see $d_1$ in Fig. \ref{fig:GPhC}). Now, if the Nyquist curve of the open loop system lies in the right side of $d_1$ the desired values for the gain margin $g_m$ and phase margin $\phi_m$ will be assured. This can be represented by a set of linear constraints thanks to the linear parameterization of the controller. Now, consider another straight line $d_2$ which is tangent to the middle of the unit circle in the sector created by $d_1$ and the imaginary axis. If we call $\omega_x$ the frequency at which the Nyquist curve intersects $d_2$, a crossover frequency greater than or equal to $\omega_x$ can be achieved by satisfying a set of linear constraints. In fact, for frequencies greater than $\omega_x$ the Nyquist curve should lie below $d_1$ and above $d_2$ while for frequencies less than $\omega_x$ it should lie below $d_2$.

Let us define the set of all points in the complex plane on the line $d$ by $f(x+iy, d) = 0$. Assume that $f(x+iy, d) < 0$ represents the half plane that excludes the critical point. Then, to find optimal controller parameters, an optimization problem like the following is used:

\newpage
 \[
\max_\rho{g}
\]
subject to:
\begin{align}
\label{eq:opGPhC}
& f(\rho^T \phi(j\omega_{ik})G_i(j\omega_{ik}),d_1)<0 \quad \mbox{for} \quad \omega_{ik}>\omega_x, \nonumber \\
& f(\rho^T \phi(j\omega_{ik})G_i(j\omega_{ik}),d_2)>0 \quad \mbox{for} \quad \omega_{ik}>\omega_x,  \\
& f(\rho^T \phi(j\omega_{ik})G_i(j\omega_{ik}),d_2)<0 \quad \mbox{for} \quad \omega_{ik} \leqslant \omega_x, \nonumber \\
& \mbox{for} \quad k=1,\ldots,N_i \:\: \mbox{and} \:\: i=1,\ldots,m. \nonumber
\end{align}
where, the objective function for minimization, $g$, can be one of the two following cases:
\begin{itemize}
\item When one wants the open-loop of the system to be close to a desired open-loop, $L_d$. In this case, $g$ would be the quadratic criterion below:
\begin{equation}
g=\sum_{i=1}^{m}\sum_{k=1}^{N_i} |L_i(j\omega_{ik},\rho)-L_d(j\omega_{ik})|^2
\label{eq:quadratic criterion}
\end{equation}
where $L_i(j\omega_{ik})=\rho^T \phi(j\omega_{ik})G_i(j\omega_{ik})$.

\item When the control objective is to optimize the load disturbance rejection of the closed-loop. This is, in general, achieved by maximizing the controller gain at low frequencies. For example, for a PID controller it corresponds to maximizing the coefficient of the integral part, i.e. $g = k_i$.
\end{itemize}

It should be noted that for all controller types including self-defined ones, one can specify an $L_d$ to minimize the criterion (\ref{eq:quadratic criterion}). But, optimizing load disturbance rejection is considered only in PID, PI, PD, and Laguerre controllers.

In many control problems a constraint on the controller gain at high frequencies can help reducing the large pick values of the control input. This can be achieved by considering a bound on the real and the imaginary part of the controller, $\rho^T\phi(j\omega_{i})$, at frequencies greater than $\omega_h$:
\begin{align}
& -K_u<Re(\rho^T \phi(j\omega_{ik}))<K_u \quad \mbox{for} \quad \omega_{ik} > \omega_h \nonumber \\
& -K_u<Im(\rho^T \phi(j\omega_{ik}))<K_u \quad \mbox{for} \quad \omega_{ik} > \omega_h 
\label{eq:Kuwh}
\end{align} 
where $Re$ and $Im$ denote, respectively, real and imaginary parts of a complex value. These linear constraints will be included in the optimization problem (\ref{eq:opGPhC}) if specified by user.

\begin{figure}
			\psfrag{A}{\scriptsize $M_m$}
			\psfrag{B}{\scriptsize $L_i(j\omega_k,\rho)$}
			\psfrag{K}{\scriptsize \textcolor{cyan}{$L_d(j\omega_k)$}}
			\psfrag{E}{\small$d(M_m,L_d(j\omega_k))$}
			\psfrag{L}{\small $R_e$}
			\psfrag{M}{\small $I_m$}
			\centering
			\includegraphics[width=0.65\columnwidth] {loopshaping.eps}
\caption{Loop shaping in Nyquist diagram by quadratic programming}
\label{fig:LS}
\end{figure}

\section{Loop shaping controller}
The performance specification can be defined by a desired open loop transfer function (or nonparametric frequency response data), $L_d(j\omega)$. It can be computed if a desired reference model $M$ is available: $L_d=M/1-M$. Typically for stable systems $L_d(s)=\omega_c/s$ would work well. Then a controller can be designed by minimizing the quadratic criterion of (\ref{eq:quadratic criterion}).

The modulus margin, the shortest distance between the Nyquist curve and the critical point, which is a better robustness indicator than the classical gain and phase margins, is considered in the loop shaping controller design method. For example, a modulus margin $M_m$ of $0.5$ is met if the Nyquist curve does not intersect a circle of radius $0.5$ centered at the critical point. This can be achieved if the Nyquist diagram is at the side of $d$, a straight line tangent to the modulus margin circle, that excludes the critical point. This constraint is linear but conservative. The conservatism can be reduced if the slop of this line changes with frequency. A good choice is a line $d(M_m,L_d(j\omega_k))$ orthogonal to the line that connects the critical point and $L_d(j\omega_k)$ and tangent to the modulus margin circle (see Fig. \ref{fig:LS}). Thus the controller is designed solving the following quadratic optimization problem:
\[
\min_\rho{\sum_{i=1}^{m}\sum_{k=1}^{N_i} |L_i(j\omega_{ik},\rho)-L_d(j\omega_{ik})|^2}
\]
subject to:
\begin{align}
\label{eq:opLS}
& f(\rho^T \phi(j\omega_{ik})G_i(j\omega_{ik}),d(M_m,L_d(j\omega_{ik})))<0 \\
& \mbox{for} \quad  k=1,\ldots,N_i \:\: \mbox{and} \:\: i=1,\ldots,m. \nonumber
\end{align}
This approach can be applied to unstable systems if $L_d$ contains the same number of unstable poles as well as the poles  on the imaginary axis of $L_i(s)$ (see \cite{KG10} for details).

\begin{figure}
			\psfrag{A}{\scriptsize $\vert W_1(j\omega_k)\vert$}
			\psfrag{B}{\scriptsize $\vert W_2(j\omega_k)L(j\omega_k,\rho)\vert$}
			\psfrag{K}{\scriptsize \textcolor{cyan}{$L_d(j\omega_k)$}}
			\psfrag{E}{\small$d(W_1(j\omega_k),L_d(j\omega_k))$}
			\psfrag{L}{\small $R_e$}
			\psfrag{M}{\small $I_m$}
			\psfrag{N}{\scriptsize $L(j\omega_k,\rho)$}
			\centering
			\includegraphics[width=0.65\columnwidth] {Hinfinity.eps}
\caption{Expression of the robust performance condition as linear or convex constraints}
\label{fig:Hinf}
\end{figure}

\section{$H_\infty$ controller}
Consider a SISO plant model with multiplicative unstructured uncertainty: $$\tilde{G}(j\omega)=G(j\omega)[1+W_2(j\omega)\Delta]$$ where $G(j\omega)$ is the plant nominal frequency function, $W_2(j\omega)$ is the uncertainty weighting frequency function, and $\Delta$ is a stable transfer function with $\|\Delta\|_\infty < 1$. In the Nyquist diagram the open loop frequency function will belong to a disk centered at $L(j\omega,\rho)$ with a radius of $|W_2(j\omega)L(j\omega,\rho)|$. This disk can be approximated by a circumscribed polygon with $n_q> 2$ vertices, such that $L_x(j\omega,\rho)=K(j\omega,\rho)G_x(j\omega)$ for $x=1,\ldots,n_q$, where
\begin{equation}
G_x(j\omega)=G(j\omega)\left[ 1+\frac{|W_2(j\omega)|}{\cos{(\pi/n_q)}} e^{j2\pi x/n_q} \right]
\end{equation}

Suppose that the nominal performance is defined as $\|W_1S\|_\infty<1$, where $S=(1+KG)^{-1}$ is the sensitivity function and $W_1$ is the performance weighting filter. This condition is satisfied if the Nyquist curve of the nominal model does not intersect the performance disk, a disk centered at the critical point with a radius of $|W_1(j\omega)|$. Therefore, the robust performance is achieved if there is no intersection between the uncertainty and performance disks \cite{DFT92} (see Fig. \ref{fig:Hinf}). This constraint can be linearized using a straight line $d(W_1(j\omega),L_d(j\omega))$ which is tangent to the performance disk and orthogonal to the line connecting the critical point and $L_d(j\omega)$ \cite{KG10}. The robust performance is met if $L_x(j\omega,\rho)$ is at the side of $d(W_1(j\omega),L_d(j\omega))$ that excludes the critical point for all $\omega$. This can be represented by the following set of linear constraints:
\begin{align}
\label{eq:rbstper}
& f(\rho^T \phi(j\omega_{k})G_x(j\omega_{k}),d(W_1(j\omega_k),L_d(j\omega_k)))<0 \\ 
& \mbox{for} \quad k=1,\ldots,N \quad \mbox{and} \quad x=1,\ldots,X. \nonumber
\end{align}
 
Then, in (\ref{eq:rbstper}), if we let $W_2=0$, it will be equivalent to the nominal performance condition: 
\begin{equation}
\label{eq:nomper}
\|W_1S\|_\infty<1
\end{equation}
and if we let $W_1=0$, the robust stability condition:
\begin{equation}
\label{eq:rbststab}
\|W_2T\|_\infty<1
\end{equation}
 will be obtained where $T$ is the complementary sensitivity function. Other than these two constraints, constraints on the weighted infinity norm of other closed loop sensitivity functions, after being linearized in a similar manner, can also be included in the optimization problem (see more details in \cite{KG10}):
\begin{equation}
\label{eq:W3W4}
\|W_3KS\|_\infty<1 \quad \mbox{and} \quad \|W_4GS\|_\infty<1 
\end{equation}
where $W_3$ and $W_4$ are corresponding weighting filters.

As a result, two different optimization problems are considered in $H_\infty$ controller design method:
\begin{enumerate} 
\item Defining a desired open-loop $L_d$ and minimizing the criterion (\ref{eq:quadratic criterion}) in the following optimization problem:
\[
\min_\rho{\sum_{i=1}^{m}\sum_{k=1}^{N_i} |L_i(j\omega_{ik},\rho)-L_d(j\omega_{ik})|^2}
\]
subject to:
$$
 \|W_1S\|_\infty<1 \, , \quad
 \|W_2T\|_\infty<1 \, , \quad
 \|W_3KS\|_\infty<1  \, , \quad
 \|W_4GS\|_\infty<1 
$$

\item Solving the following optimization problem:
\begin{equation}
\begin{split}
& \hspace{4.5cm}\min{\gamma} \\
&\mbox{ subject to: } \\
&\| \lambda_1 |W_1S| + \lambda_2 |W_2 T| + \lambda_3 |W_3 KS| + \lambda_4 |W_4 GS|  \|_\infty < \gamma
\end{split}
\label{gamma_optim}
\end{equation}
where $\lambda=[\lambda_1, \: \lambda_2, \: \lambda_3, \: \lambda_4]$ is a vector of positive coefficients determining the importance of each constraint. If $\lambda_i=0$ and $W_i$ is defined, then the constraint corresponding to $W_i$ will also be considered. For example if $\lambda=[1 \: 0 \:1 \:0]$ and $W_1$, $W_2$ and $W_3$ are defined the following optimization problem is solved:
\[
\min{\gamma}
\]
subject to:
$$\|  |W_1S| + |W_3 KS|  \|_\infty < \gamma$$
$$
 \|W_2T\|_\infty<1 
$$
If $\lambda=[0 \: 0 \:0 \:0]$ an upper bound on all weighted sensitivity function will be minimize:
\[
\min{\gamma}
\]
subject to:
$$\| W_1S \|_\infty < \gamma  \, , \,  \|W_2T\|_\infty< \gamma  \, , \,   \| W_3KS \|_\infty < \gamma   \, , \,  \|W_4GS\|_\infty< \gamma$$

As stated before, all these constraints are used in their linearized form and  $\gamma$ is minimized by a bisection algorithm to reach the best performance possible in terms of the above constraints. 

\end{enumerate}

\section{MIMO controller}
The performance specifications for SISO systems can also be used for designing MIMO controllers if the open loop system is decoupled. The main idea is to design a MIMO decoupling controller such that the open-loop transfer matrix $L(j\omega)$ becomes diagonally dominant. For this reason a diagonal desired open loop transfer matrix $\textbf{L}_d$ is considered and the following quadratic criterion is minimized:
\begin{equation}
\label{eq:L-LD}
J(\rho)=\sum_{i=1}^{m}\sum_{k=1}^{N_i} \| \textbf{L}_i(j\omega_{ik},\rho)- \textbf{L}_d(j\omega_{ik}) \|_F
\end{equation}
where $F$ stands for the Frobenius norm.

MIMO controllers presented by a matrix of transfer functions are considered where each element $K_{ij}$ of the matrix should be linearly parameterized, i.e., $K_{ij}=\rho_{ij}^T\phi_{ij}$. The controller parameters are obtained by minimizing $J(\rho)$ under some constraints to meet the SISO specifications for each diagonal element. 

In MIMO systems, besides the performance constraints, there are other constraints implying the stability of MIMO systems that should be considered. In fact, because the closed-loop system will not be completely diagonal, the stability of dominant loops will not guarantee the stability of the MIMO system. However, a stability condition can be obtained based on Gershgorin bands (see \cite{GKL10b}):
\begin{align}
\label{eq:Gersh}
& r_q(\omega_k,\rho)\big| [1+L_{dq}(j\omega_k)]\big|-Re\big([1+L_{dq}(-j\omega_k)][1+L_{qq}(j\omega_k,\rho)]\big)<0 \nonumber \\
& \mbox{for} \quad q=1,\ldots,n_o \quad  \mbox{and} \quad k=1,\ldots,N 
\end{align}
where
\[
r_q(\omega,\rho)=\sum_{p=1,p\neq q}^{n_o} |L_{pq}(j\omega,\rho)|,
\]
$n_o$ is the number of the outputs of the system, and $L_{dq}$ is the $q^{th}$ diagonal element of $L_d$. This constraint is written for one model. It will be considered for all models when tackling multi-model systems.

In summary, in MIMO systems, since the method is based on decoupling, always the criterion (\ref{eq:L-LD}) is minimized. Hence, for every diagonal element of the open loop matrix ($\textbf{L}=\textbf{GK}$), an $L_d$ should be specified. ($\textbf{L}_d$ is a diagonal matrix with these $L_d$'s as its diagonal elements.) Also all the previously explained performance conditions (GPhC, loop shaping, and $H_\infty$) will be applied on the diagonal elements of the open loop transfer matrix. The stability conditions of (\ref{eq:Gersh}) will be added to the other performance constraints. 

\section{Gain-scheduled controller}

All presented robust controller design methods for systems with multimodel uncertainty can be extended to designing
gain-scheduled controllers. Suppose that each model $G_i$ is associated to a value of a scheduling parameter vector $\theta$, which is measured in real time. The controller parameters can be polynomial functions of $\theta$ and be computed by the optimization algorithm. Then, in all the previous constraints, we would place:

\[
\rho=M\bar{\theta}_l
\]
where
\begin{align}
\label{eq:GS}
& M=\left[ \begin{array}{cccc}
(\rho_{1,n_p})^T & \cdots & (\rho_{1,1})^T & (\rho_{1,0})^T \\ [.25 cm]
 \vdots & \ddots & \vdots & \vdots \\ [.25 cm]
(\rho_{n_\rho,n_p})^T & \cdots & (\rho_{n_\rho,1})^T & (\rho_{n_\rho,0})^T \end{array} \right] \\
& \mbox{and} \quad \bar{\theta}_l=[\theta_l^{n_p}\:  \ldots \: \theta_l \: \overrightarrow{1}]^T. \nonumber 
\end{align}
where $n_p$ is the order of polynomials describing controller parameters and $n_\rho$ is the dimension of $\rho$ which is the same as the dimension of the vector of the basis functions $\phi$.

For instance, for a PID controller ($n_\rho=3$) with a scalar scheduling parameter and the vector $\rho$ as a second order polynomial of $\theta$ ($n_p=2$) we will have a parametrization like this:
\begin{equation}
\label{eq:GSPID}
\rho(\theta)=\left[ \begin{array}{ccc}
k_{p_2} & k_{p_1} & k_{p_0} \\ [.25 cm]
k_{i_2} & k_{i_1} & k_{i_0} \\ [.25 cm]
k_{d_2} & k_{d_1} & k_{d_0} \end{array} \right] \: \left[ \begin{array}{c}
\theta^2 \\ [.25 cm]
 \theta \\ [.25 cm]
1 \end{array} \right] 
\end{equation}
For more details about the gain-scheduled controller design see \cite{KKL07e}.

\chapter{Toolbox commands}
The procedure of design comprises three steps. First the type (or structure) of the controller should be determined. Then the desired performance characteristics are specified, and finally a controller with the desired type and performance is designed. In the following comes a description of these three steps with corresponding commands.

\section{Determining controller structure}
The first step of design is determining the desired controller type. By defining the controller type in fact the vector of basis transfer functions $\phi$ is specified. In the following command the controller type and subsequently the vector $\phi$ are specified by the user.
\begin{lstlisting}
<<<<<<< HEAD
phi = conphi (ConType , ConPar , CorD , F, ConStruc) 
=======
phi = conphi (ConType , ConPar , CorD , F, ConStruc, ConOpt) 
>>>>>>> statespace
\end{lstlisting}
%\begin{table}
%\centering
%\caption{Available controller structures and the corresponding parameters to be defined by user} \smallskip
%%\newcolumntype{R}{>{\raggedleft\arraybackslash}X}
%\begin{tabular}{lrc}
%\hline \hline \noalign{\smallskip}
% ConType  & \multicolumn{2}{c}{ConPar} \\  [.3 cm] \hline \hline \noalign{\smallskip}
%  \multirow{2}{*}{'PID'} & Continuous & \texttt{$\tau$} \\ 
%  & Discrete & \texttt{[Ts, $\tau$]} \\ [.3 cm] \hline \noalign{\smallskip}
%  \multirow{2}{*}{'PI'} & Continuous & \texttt{[]}  \\
%  & Discrete & \texttt{Ts} \\ [.3 cm] \hline \noalign{\smallskip}
%  \multirow{2}{*}{'PD'} & Continuous & \texttt{$\tau$} \\
%  & Discrete & \texttt{[Ts, $\tau$]}  \\ [.3 cm] \hline \noalign{\smallskip}
%  \multirow{2}{*}{'Laguerre'} & Continuous & \texttt{[$\xi$, n]} \\
%  & Discrete & \texttt{[Ts, a, n]}  \\ [.3 cm] \hline \noalign{\smallskip}
%  \multirow{2}{*}{'generalized'} & Continuous & \texttt{[$\xi_1$, $\xi_2$, $\ldots$, $\xi_n$]} \\
%  & Discrete & \texttt{[Ts, $\xi_1$, $\xi_2$, $\ldots$, $\xi_n$]}  \\ [.3 cm] \hline
%\hline
%\end{tabular}
%\label{tab:first command}
%\end{table}

\begin{description}
\item[\texttt{ConType}] is a string representing the desired controller type. It is not case sensitive when being defined. it can be:

  \begin{tabular}{ll}
       {\tt `PID'}     &    For PID controller \\
      {\tt `PD' }       & For PD controller\\
       {\tt `PI' }        &  For PI controller\\
      {\tt  `Laguerre' }  &  For Laguerre basis function\\
      {\tt  `Generalized' }  & For Generalized basis function\\
    {\tt `UD'}        &   For user defined structure 
\end{tabular}

\item[\texttt{ConPar}]  is a scalar or a vector of parameters for the chosen controller type.
For continuous-time PID or PD controller {\tt ConPar} can be the time constant
       of the derivative part (if it is not specified a default value equal to $1.2/\omega_{\max}$ will be computed, where $\omega_{\max}$ is the maximum value of the vector $\omega$).

For discrete time PI, PD and PID controllers, {\tt ConPar} specifies the sampling period.

For continuous-time  Laguerre basis function, {\tt ConPar} is $[\xi \quad n]$ where $\xi$ defines the parameter of 
       Laguerre basis and $n$ is its order. For discrete-time Laguerre basis function, {\tt ConPar} is 
       $[T_s \quad a \quad n]$ where $T_s$ is the sampling period, $a$ the parameter of Laguerre basis and $n$ its order.
       
       For continuous-time generalized basis function, {\tt ConPar} is $\xi$, a $n$-th
       dimensional vector containing the parameters of the generalized basis
       function. For discrete-time generalized basis function, {\tt ConPar} is
       $[T_s \quad \xi]$.

       For user defined structure, {\tt ConPar} is a column vector of stable
       transfer functions.

\item[\texttt{CorD}] is either \texttt{`s'} or \texttt{`z'} showing, respectively, that the controller is in continuous- or discrete-time. If not mentioned, the continuous-time case will be considered. 

\item[\texttt{F}] is a transfer function by which the vector $\phi$ is multiplied. For example one can multiply a factor of integral $\frac{1}{s}$ to PID basis functions to get: $\phi=\frac{1}{s} \times [1\:\: \frac{1}{s} \:\: \frac{s}{1+\tau s}]^T$.

<<<<<<< HEAD
\item[\texttt{ConStruc}] is either \texttt{'LP'} for a linearly parameterized controller (default) or \texttt{'TF'} for a transfer function (rational) controller.

=======
\item[\texttt{ConStruc}] is either \texttt{'LP'} for a linearly parameterized controller (default) or \texttt{'SS'} for a state-space controller.

\item[\texttt{ConOpt}] for a state space controller: is a cell containing a string \texttt{'B'} or \texttt{'C'} and the desired B or C matrix of the controller. By default, C = [1 0 0 0 ...].
>>>>>>> statespace
    

\end{description}


\subsection{Example 1} A continuous-time PID controller with $\tau=0.1$:

\begin{lstlisting}
phi = conphi ('PID' , 0.1) 

phi.phi
 
Transfer function from input to output...
 #1:  1
 
      1
 #2:  -
      s
 
          s
 #3:  ---------
      0.1 s + 1      
\end{lstlisting}

\subsection{Example 2} A discrete-time PI controller with sampling time of 0.05 seconds:

\begin{lstlisting}
phi = conphi ('PI' , 0.05 , 'z') 

phi.phi
 
Transfer function from input to output...
 #1:  1
 
        z
 #2:  -----
      z - 1
 
Sampling time: 0.05
\end{lstlisting}

\subsection{Example 3} A continuous-time PID controller with $\tau=0.1$ multiplied by an integrator:

\begin{lstlisting}
s=tf('s');  F=1/s;
phi = conphi ('PID' , 0.1 , 's' , F) 

phi.phi
 
Transfer function from input to output...
      1
 #1:  -
      s
 
       1
 #2:  ---
      s^2
 
           s
 #3:  -----------
      0.1 s^2 + s    

\end{lstlisting}

\subsection{Example 4} A continuous-time 3rd-order Laguerre basis with $\xi=1$:

\begin{lstlisting}
phi = conphi ('Laguerre',[1,3])

phi.phi
 
Transfer function from input to output...
 #1:  1
 
      1.414
 #2:  -----
      s + 1
 
      1.414 s - 1.414
 #3:  ---------------
       s^2 + 2 s + 1
 
      1.414 s^2 - 2.828 s + 1.414
 #4:  ---------------------------
         s^3 + 3 s^2 + 3 s + 1
         
\end{lstlisting}

\subsection{Example 5} A continuous-time, user defined vector of basis functions:

\begin{lstlisting}
s = tf ('s');
phi = conphi('ud',[1 ; 1/s ; s/(s^2+2*s+1)]);

phi.phi
 
Transfer function from input to output...
 #1:  1
 
      1
 #2:  -
      s
 
            s
 #3:  -------------
      s^2 + 2 s + 1
  \end{lstlisting}

<<<<<<< HEAD
  \subsection{Example 6} A continuous-time 4th order Laguerre basis with $\xi=20$ in a transfer function structure, with an integrator (phi.fs is the inverse of the fixed part of the controller)

  \begin{lstlisting}
s = tf('s'); xi = 20; n = 4;
phi = conphi('Lag',[xi n],'s',1/s,'tf');


phi.phi

ans =
 
  From input to output...
   1:  1
 
       6.3246
   2:  ------
       (s+20)
 
       6.3246 (s-20)
   3:  -------------
         (s+20)^2
 
       6.3246 (s-20)^2
   4:  ---------------
          (s+20)^3
 
       6.3246 (s-20)^3
   5:  ---------------
          (s+20)^4
 
Continuous-time zero/pole/gain model.


phi.fs

ans =
 
  s
 
Continuous-time transfer function.
\end{lstlisting}

=======
  \subsection{Example 6} A continuous time PID controller defined in state space

\begin{lstlisting}
tau = 0.1;
phi = conphi('PID',tau,'s',[],'SS');

phi.phi

 
From input to output...
    1
1:  -
    s
 
        1
2:  --------
     s (s+10)
 
3:  1


phi.par.A

         0    1.0000
         0  -10.0000


phi.par.C

     1     0
         
\end{lstlisting}


\subsection{Example 7} A discrete time Laguerre controller defined in state space with the B matrix given

\begin{lstlisting}
Ts = 0.02; n = 4;
z = tf('z');

phi = conphi('Laguerre',[Ts 0.1 n],'z',[],'SS',{'B',[1 0; 1 0; 0 1; 0 1]});


phi

	[1x1 struct]    [1x1 struct]

    
phi{1}.phi
    
  From input to output...
       (z-0.2048) (z^2 - 0.1952z + 0.02002)
   1:  ------------------------------------
                    (z-0.1)^4
 
       (z+1) (z^2 - 0.4z + 0.06)
   2:  -------------------------
               (z-0.1)^4
 
       (z-0.4) (z+1)
   3:  -------------
         (z-0.1)^4
 
         (z+1)
   4:  ---------
       (z-0.1)^4
 
   5:  1
 
Sample time: 0.02 seconds
Discrete-time zero/pole/gain model.

\end{lstlisting}


>>>>>>> statespace
After defining our desired controller structure, we shall proceed to the next step: specifying performance characteristics.



\section{Determining control performance}
The desired control performance attributes of the system are determined by the following command:
\begin{lstlisting}
per = conper (PerType , par , Ld) 
\end{lstlisting}
%\begin{table}
%  \centering 
%  \caption{The parameters corresponding to each type of performance} \smallskip \smallskip
%  \begin{tabular}{ll}
%  {\tt PerType} & {\tt par}  \\ [0.3 cm] \hline \hline \noalign{\smallskip} \noalign{\smallskip} \noalign{\smallskip} 
%   \texttt{'GPhC'}  & \texttt{[gm , pm , wc , Ku , wh]}  \\ [.5 cm]
%   \texttt{'LS'}& \texttt{[Mm , Ku , wh]}  \\ [0.5 cm]
%   \multirow{5}{*}{\texttt{'Hinf'}} & \texttt{par.W1} \\
%   & \texttt{par.W2}\\
%   & \texttt{par.W3}\\
%   & \texttt{par.W4}\\
%   & \texttt{par.lambda} \\ [.3 cm]
%\hline \hline
%\end{tabular}
%  \label{tab:second command}
%\end{table}

\begin{description}
<<<<<<< HEAD
\item[\texttt{PerType}] is a string specifying the desired performance of the system. It can be \texttt{`GPhC'}, \texttt{`LS'} or \texttt{`Hinf'}. For rational controllers (with \texttt{`TF'} structure), only \texttt{`Hinf'} can be used.
=======
\item[\texttt{PerType}] is a string specifying the desired performance of the system. It can be \texttt{`GPhC'}, \texttt{`LS'} or \texttt{`Hinf'}. 
>>>>>>> statespace

\begin{description}
\item{\texttt{`GPhC'}} 
stands for the case when one wants to determine desired values for gain margin, phase margin, and crossover frequency. 


\item{\texttt{`LS'}} stands for Loop Shaping controller. In this method the distance between $L=GK$ (open-loop transfer function)
              and $L_d$ (the desired one) in one thousand frequencies linearly spaced between $\omega_{\min}$ and $\omega_{\max}$ is
              minimized. A lower bound on the Modulus margin (the inverse of the infinity norm of the sensitivity function) is also
              guaranteed.



\item{\texttt{`Hinf'}} In this method the distance between $L=GK$ and $L_d$ is minimized under 
<<<<<<< HEAD
             some $H_\infty$ constraints on the weighted closed-loop sensitivity functions. For rational controllers, $L_d$ is not used $\gamma$ is minimized. 
=======
             some $H_\infty$ constraints on the weighted closed-loop sensitivity functions.
>>>>>>> statespace
             
 \end{description}            
             
           
             
 \item[\texttt{par}] is a structure that contains all the data specified by the user in this command.
 
 For the {\texttt{`GPhC'}} method \texttt{par} is a vector containing the lower bounds of gain margin $g_m$, phase margin $\varphi_m$, crossover frequency $\omega_c$, and an upper bound for the controller gain $K_u$ which may be applied at frequencies higher than $\omega_h$. 
The crossover frequency $\omega_c$, $K_u$ and $\omega_h$ are optional values. If they are not assigned, no lower bound for the crossover frequency and  no upper bound for the controller gain in high frequencies is considered.
If \texttt{Ld} is specified, the quadratic criterion (\ref{eq:quadratic criterion}) will be minimized; otherwise, the controller gain at low frequencies will be maximized. It should, however, be noted that maximizing controller gain at low frequencies will be done in PID, PI, PD, and Laguerre controllers. In \texttt{`generalized'} type and in \texttt{`user defined'} basis functions, an \texttt{Ld} should be specified unless no optimization is performed and only a feasible solution is given.
     
 For the {\tt `LS'} method, \texttt{par} is a vector containing modulus margin $M_m$, and (if desired) $K_u$ and $\omega_h$. In loop shaping, the objective is to force open loop to act like a desired open loop function. So a desired open loop function ($L_d$) should always be specified by the user.    
     
             
For the {\tt `Hinf'} method, {\tt par} is a cell, W, containing up to four weighting
filters $W_{1}, W_{2}, W_{3}$ and $W_{4}$. The following constraints are applied:
 $$
 \|W_1S\|_\infty<1 \, , \quad
 \|W_2T\|_\infty<1 \, , \quad
 \|W_3KS\|_\infty<1  \, , \quad
 \|W_4GS\|_\infty<1 
$$
       where $S=(1+GK)^{-1}$ and $T=1-S$ are respectively sensitivity and complementary 
       sensitivity functions. $W_{i}$ can be any LTI type model or frequency domain model 
       (e.g. `frd' model). The distance between $L=GK$ and $L_d$ is minimized.


\item[\texttt{Ld}] is a desired open loop which can be a parametric transfer function or a nonparametric \texttt{frd} object containing frequency response data over a frequency vector. It should contain the
     poles on the stability boundary of the plant model and the controller. It should also 
     satisfies the Nyquist stability criterion.



\end{description}

\subsection{Examples}

\begin{description}

\item[{\tt GPhC :}] A lower bound of 2 for gain margin and 60 degree for phase margin is considered in the following command:
\begin{lstlisting}
performance=conper('GPhC', [2 , 60]);
\end{lstlisting}
For PID controllers as well as Laguerre basis functions the low frequency gain of the controller will be maximized by the linear programming approach. A lower bound of 3 rad/s for the crossover frequency 	can be added by:
\begin{lstlisting}
performance=conper('GPhC', [2 , 60, 3]);
\end{lstlisting}
The hard constraint on the crossover frequency can be replaced by defining {\tt Ld=3/s} and minimizing {\tt F(L-Ld)} by the quadratic programming approach as follows:
\begin{lstlisting}
performance=conper('GPhC', [2 , 60], 3/s);
\end{lstlisting}	
({\tt F} will be defined later on). If the controller gain at high frequencies is too large, say greater than 20 such that the control input becomes saturated, it can be limited to, say 10, for frequencies greater than 30 rad/s by the following command:
\begin{lstlisting}
performance=conper('GPhC', [2 , 60, 0, 10, 30], 3/s);
\end{lstlisting}  
Suppose that a robust controller should be designed that guarantees a gain margin of 2 and a phase margin of 45  for two models $G_1$ and $G_2$. Assume also that the lower bound of the crossover frequency for the first model is 1 rad/s and for the second model is 5 rad/s. The performances can be defined as follows:
\begin{lstlisting}
MMper{1}=conper('GPhC', [2 , 45, 1]);
MMper{2}=conper('GPhC', [2 , 45, 5]);
\end{lstlisting}

\item[{\tt LS :}] A modulus margin of 0.6 and a desired open-loop transfer function of {\tt Ld=10/s} can be defined as control performance by the following command:
\begin{lstlisting}
LSperformance=conper('LS', 0.6, 10/s);
\end{lstlisting}  
An upper bound of 30 for frequencies greater than 100 rad /s can be considered by:
\begin{lstlisting}
LSperformance=conper('LS', [0.6 , 30 , 100], 10/s);
\end{lstlisting} 

\item[{\tt Hinf :}] Consider performance weighting filters $W_1$, multiplicative uncertainty filter $W_2$ and input sensitivity weighting filter $W_3$ in an $H_\infty$ controller design problem then the following commands can be used:
\begin{lstlisting}
W{1}=W1;W{2}=W2;W{3}=W3;
Ld=W{1}-1;
HinfPer=conper('Hinf', W, Ld);
\end{lstlisting} 
Note that if we consider that the desired sensitivity function $S_d$ is close to the inverse of $W_1$, a good choice for $L_d$ is $W_1-1$ (because $S_d^{-1}=1+L_d$).

\end{description}



\section{Controller Design}
After gathering the required data from user, the controller is designed by the following command:

\begin{lstlisting}  
K = condes (G , phi , per , options)
\end{lstlisting}

\begin{description}
<<<<<<< HEAD
\item[\texttt{G}] is a cell, i. e., \texttt{G\{1\}}, \texttt{G\{2\}}, \dots, \texttt{G\{m\}} represent SISO or MIMO models $G_1$, $G_2$, $\ldots$, $G_m$. In case there is just one model, define it as \texttt{G\{1\}} or simply \texttt{G}.
    For a \texttt{`TF'} (transfer function/rational) controller structure, G can be given as a cell of cells where \texttt{N\{j\}=G\{j\}\{1\}} and \texttt{M\{j\}=G\{j\}\{2\}}. The j-th plant model is equal to \texttt{N\{j\} M\{j\}$^{-1}$}.
=======
\item[\texttt{G}] is a cell, i. e., \texttt{G\{1\}}, \texttt{G\{2\}}, \dots, \texttt{G\{m\}} represent SISO or MIMO models $G_1$, $G_2$, $\ldots$, $G_m$. In case there is just one model, define it as \texttt{G\{1\}} or simply \texttt{G}. 
>>>>>>> statespace
Nonparametric models should be defined as an \texttt{frd} or  \texttt{idfrd} object (type \texttt{doc frd} to see details about creating or converting your model to an \texttt{frd} object).

\item[\texttt{phi}] is the output of the first command. The command \texttt{condes} designs a controller of the specified type to meet the performance criteria implied by \texttt{per}. For MIMO systems \texttt{phi} is an $n_i \times n_o$ cell where $n_i$ is the number of inputs and $n_o$ is the number of outputs of the system. For example \texttt{phi\{p,q\}} is the vector of basis functions $\phi_{pq}$ for the element of row \texttt{p} and column \texttt{q} of the controller matrix $K$. So each element of the controller matrix may have a different vector of basis functions, or different structure. They should be defined separately by the first command, for example in a loop. Should you specify one controller type or structure for all the elements of the controller matrix, just simply enter one \texttt{phi} (not a cell) in the \texttt{condes} command. 

{\it Example :} Consider a $2 \times 2$ MIMO controller where the diagonal elements are Laguerre basis functions of order 3 with an integral action and the off diagonal elements are PI controllers, Then the following commands should be used:

\begin{lstlisting}
s = tf ('s');
phi{1,1} = conphi('Laguerre',[1 3],'s',1/s);
phi{1,2} = conphi('PI');
phi{2,1} = conphi('PI');
phi{2,2} = conphi('Laguerre',[1 3],'s',1/s);
\end{lstlisting}

\item[\texttt{per}] is a cell. \texttt{per\{1\}}, \texttt{per\{2\}}, \dots, \texttt{per\{m\}} contain the desired performance characteristics of, respectively, \texttt{G\{1\}}, \texttt{G\{2\}}, \dots, \texttt{G\{m\}}. When you want to apply one performance criterion for all models, simply just enter \texttt{per} (you don't need to define a cell). For MIMO systems, \texttt{per} is a cell. \texttt{per\{i\}\{q\}} ($i=1,\ldots,m$ and $q=1,\ldots,n_o$) contains the performance characteristics of the $q^{th}$ diagonal element of the open loop \texttt{L\{i\}} ($L_i=G_i \times K$), which should be defined by the  command \texttt{conper}. In MIMO systems the objective function (\ref{eq:L-LD}) will be minimized (the aim is to design a controller to decouple the system as much as possible). Hence \texttt{Ld} must be defined in \texttt{per} for every diagonal element of every open loop system. If your performance characteristics differ for the different diagonal elements of the open loop matrix, yet are the same for every model, you can simply define \texttt{per\{1\}}, \texttt{per\{2\}}, \dots, \texttt{per\{$n_o$\}} where \texttt{per\{q\}} contains the performance characteristics of the $q^{th}$ diagonal element of the open loop \texttt{L\{i\}} for $i=1,\ldots,m$. If your desired performance characteristics are the same for all the diagonal elements in all of the models, you can simply enter one \texttt{per} (not a cell) in the \texttt{condes} command to be applied to all of them.

{\it Example :} Suppose that a robust $2 \times 2$ MIMO controller should be designed for three MIMO models $G_1$, $G_2$ and $G_3$. Assume that the desired open-loop transfer function for the first output is $2/s$ and for the second output $10/s$ with a modulus margin of 0.5 for all models then the performance is defined as follows:
\begin{lstlisting}
MIMOper{1}=conper('LS', 0.5, 2/s);
MIMOper{2}=conper('LS', 0.5, 10/s);
\end{lstlisting}  
Now suppose that the lower bound on the modulus margin for the first model is 0.3, for the second model is 0.4 and for the third one is 0.5. In this case the performance is defined by:
\begin{lstlisting}
MIMOper{1}{1}=conper('LS', 0.3, 2/s);
MIMOper{1}{2}=conper('LS', 0.3, 10/s);
MIMOper{2}{1}=conper('LS', 0.4, 2/s);
MIMOper{2}{2}=conper('LS', 0.4, 10/s);
MIMOper{3}{1}=conper('LS', 0.5, 2/s);
MIMOper{3}{2}=conper('LS', 0.5, 10/s);
\end{lstlisting} 

\item[\texttt{options}] is a structure whose fields can be set by a specific command that will be explained in the next subsection.

\end{description}


When determining an \texttt{Ld} in \texttt{per}, (which is compulsory in \texttt{`LS'} and \texttt{`Hinf'} cases but optional in \texttt{`GPhC'} case) the objective of design will be minimizing the difference between the system open loop and the desired open loop \texttt{Ld} which is a quadratic objective function in terms of controller parameters $\rho$. The constraints in all of the three cases are linear in terms of $\rho$. Hence we have a quadratic programming problem which can be solved by the solver \texttt{quadprog}.

%In \texttt{`GPhC'} case, however, one can leave \texttt{Ld} unspecified intending to solve one of the optimization problems (\ref{eq:opGPhC}). The objective function and the constraints of these optimization problems are linear in terms of $\rho$. Then the solver \texttt{linprog} would be used to find optimal controller.

In MIMO systems the Gershgorin stability conditions are applied on the controller design. These constraints are convex in terms of $\rho$, but can be linearized without conservatism for systems with two outputs. For systems with more than two outputs a convex optimization solver is required. The users should instal Yalmip package and a convex optimization solver (e.g. \texttt{sdpt3}). The use of Yalmip and convex optimization solvers increase the execution time but can be used if the users do not have the optimization toolbox of Matlab.  However, regarding the fact that the Gershgorin bands constraints are sometimes conservative, one can skip applying them (see options section below). In this case, the standard solver \texttt{quadprog} will be used for solving the resulting quadratic optimization problem.

\subsection{Examples}

\subsubsection{Controller design with {\tt GPhC} performance for a single model system}
Design a PID controller for the following first order model with delay:
$$G=\frac{e^{-s}}{(s+1)^3}$$
The objective is to ensure: Gain margin = 2, Phase margin = $60^{\circ}$, Crossover frequency = 0.08 rad/s. 
\begin{lstlisting}
s=tf('s');
G=exp(-s)/(s+1)^3;

phi=conphi('PID');
per=conper('GPhC',[2,60,.08]);

K=condes(G,phi,per)
 
174.0174 (s^2 - 0.09479s + 0.04465)
-----------------------------------
            s (s+83.33)  
            
\end{lstlisting}

\subsubsection{Controller design with {\tt GPhC} performance for a multi-model system}
Consider a system that has three different models in three operating points as follows:
$$G_1=\frac{4e^{-3s}}{(10s+1)} \qquad G_2=\frac{e^{-5s}}{(s^2+14s+7.5)}  \qquad G_3=\frac{2e^{-s}}{(20s+1)}$$
The objective is to design a robust PID controller to ensure a gain margin of 3, phase margin of 60 for all models with different desired crossover frequencies as follows: $\omega_{c_1}$=0.2 rad/s (for model 1), $\omega_{c_2}$=0.01 rad/s (for model 2), $\omega_{c_3}$=0.07 rad/s (for model 3).

\begin{lstlisting}
s=tf('s');
G{1}=exp(-3*s)*4/(10*s+1);
G{2}=exp(-5*s)/(s^2+14*s+7.5);
G{3}=exp(-s)*2/(20*s+1);

phi=conphi('PID',.05);

per{1}=conper('GPhC',[3,60,.2]);
per{2}=conper('GPhC',[3,60,.01]);
per{3}=conper('GPhC',[3,60,.07]);

K=condes(G,phi,per)

13.9441 (s+0.1735) (s+0.5121)
-----------------------------
          s (s+20)
          
\end{lstlisting}

\subsubsection{Controller design with {\tt LS} performance for a multivariable system}
Consider a MIMO model given by:
$$ G(s)=\left[ \begin{array}{cc}
\frac{5e^{-3s}}{4s+1}  &  \frac{2.5e^{-5s}}{15s+1} \\[.25 cm]
\frac{-4e^{-6s}}{20s+1}  &  \frac{e^{-4s}}{5s+1} 
\end{array} \right] $$
The objective is to design a MIMO PI controller to achieve a modulus margin of 0.5 for the diagonal open-loop systems and a desired open-loop transfer function:
$$L_d(s)=\left[ \begin{array}{cc}
\frac{1}{30 s}  &  0 \\[.25 cm]
0  &  \frac{1}{30 s} 
\end{array} \right] $$

\begin{lstlisting}
s=tf('s');
G=[5*exp(-3*s)/(4*s+1) 2.5*exp(-5*s)/(15*s+1); -4*exp(-6*s)/(20*s+1) exp(-4*s)/(5*s+1)];

phi=conphi('PI'); 

per=conper('LS',0.5,1/(30*s)); 

K=condes(G,phi,per)

Zero/pole/gain from input 1 to output...
      0.029928 (s+0.07842)
 #1:  --------------------
               s
 
      0.041268 (s+0.1954)
 #2:  -------------------
               s
 
Zero/pole/gain from input 2 to output...
      -0.007108 (s+0.7555)
 #1:  --------------------
               s
 
      0.15087 (s+0.07343)
 #2:  -------------------
               s
\end{lstlisting}

<<<<<<< HEAD
=======
\subsubsection{Controller design in state space with {\tt LS} performance for a multivariable system}
Consider the same MIMO model as in the previous example.
The objective is to design a MIMO PID controller in state space to achieve the same performance as in the previous example (modulus margin of 0.5, open-loop transfer function of $\frac{1}{30s}$).

\begin{lstlisting}
s=tf('s');
G=[5*exp(-3*s)/(4*s+1) 2.5*exp(-5*s)/(15*s+1); -4*exp(-6*s)/(20*s+1) exp(-4*s)/(5*s+1)];

phi=conphi('pid',[],'s',[],'ss'); 
per=conper('LS',0.5,1/(30*s)); 
opts = condesopt('yalmip','off','gbands','off');

K=condes(G,phi,per,opts)


K =
 
  a = 
           x1      x2
   x1       0       1
   x2       0  -8.333
 
  b = 
             u1        u2
   x1  -0.07789   -0.2858
   x2    0.6801      2.37
 
  c = 
       x1  x2
   y1   1   0
   y2   1   0
 
  d = 
            u1       u2
   y1  0.03529  0.01411
   y2  0.03274   0.1328
 
Continuous-time state-space model.

\end{lstlisting}

>>>>>>> statespace
\subsubsection{Controller design with {\tt Hinf} performance}
Consider the following system:
$$G(z)=\frac{z-0.2}{z^3-1.2z^2+0.5z-0.1}$$
with sampling period $Ts=1 s$. Compute a third-order discrete-time controller with integral action that satisfies $\| W_1 S\|_\infty < 1$ and has a bandwidth of 0.2 rad/s, where
$$W_1(z)=\frac{0.4902(z^2-1.0432z+0.3263)}{z^2-1.282z+0.282}$$
We proceed as follows:
\begin{lstlisting}
Ts=1;
z=tf('z',Ts);
s=tf('s');

G=(z-0.2)/(z^3-1.2*z^2+0.5*z-0.1);
W{1}=0.4902*(z^2-1.0432*z+0.3263)/(z^2-1.282*z+0.282);

Ld=0.2/s;

phi=conphi('Laguerre',[Ts 0 3],'z',z/(z-1));

per=conper('Hinf',W,Ld);

K=condes(G,phi,per)

0.15968 (z+0.0614) (z^2 - 0.9455z + 0.2318)
-------------------------------------------
                 z^2 (z-1)


\end{lstlisting}

<<<<<<< HEAD

\subsubsection{Rational controller design}
Consider the following system:
$G(z) = \frac{z-0.186}{z^3 - 1.116z^2 + 0.465 z - 0.093}$
with sampling period $Ts = 1 s$. Compute a fourth-order stabilizing controller with an integrator that minimizes $\| W_1 S\|_\infty$, where
$W_1(z) = \frac{0.4902 \left( z^2 - 1.0431 z + 0.3263\right)}{(z-1)(z-0.282)}$

We proceed as follows:
\begin{lstlisting}
Ts = 1; n = 4; xi = 0;
s = tf('s');
z = tf('z',Ts);

G = (z-0.186)/(z^3 - 1.116*z^2 + 0.465*z - 0.093);
W{1} = 0.4902 *( z^2 - 1.0431* z + 0.3263)/((z-1)*(z-0.282));

phi = conphi('Laguerre',[Ts xi n],'z',z/(z-1),'TF');
per = conper('Hinf',W);
opts = condesopt('gamma',[0.1 1 1e-3]);

K = condes(G,phi,per,opts)

  0.56039 z (z-0.5241) (z+0.215) (z^2 - 0.1235z + 0.3811)
  -------------------------------------------------------
   (z+0.4187) (z+0.8764) (z-1) (z^2 - 0.006408z + 0.365)
 
\end{lstlisting}


=======
>>>>>>> statespace
\section{Controller design options}

The controller design options are defined by the following command:
\begin{lstlisting}
options = condesopt ('param1',value1,'param2',value2,...)
\end{lstlisting}

%\begin{table}
%  \centering 
%  \caption{Toolbox options} \smallskip \smallskip
%  \begin{tabular}{l l l}
%\hline \hline \noalign{\smallskip} \noalign{\smallskip} 
%  Option & Default value & Description  \\ [0.3 cm] \hline \hline \noalign{\smallskip} \noalign{\smallskip} \noalign{\smallskip} 
%   \texttt{w} & \texttt{[]} & \texttt{w} is a cell. \texttt{w\{i\}} is the vector of frequencies over which $G_i(j\omega)$ is known.\\ [.3 cm]
%   \texttt{MIMOstability}& \texttt{true} & Shows that Gershgorin stability conditions for the MIMO system will be applied. Set it to logical 0 (false) to prevent applying them. \\ [0.3 cm]
%   \texttt{beta} & \texttt{20} & The angle of line $d_2$ with the real axis in degrees \\ [0.3 cm]
%   \texttt{np} & \texttt{[]} & The degree of polynomials describing the gain-scheduled controller parameters \\ [.3 cm]
%   \texttt{GSpar} & \texttt{[]} & A vector (or matrix if we have more than one scheduling parameter) of the scheduling parameter values in designing a gain-scheduled controller\\
%   Other solver options & \texttt{[]} & All available options for the solver in use.\\[.3 cm]  
%\hline \hline
%\end{tabular}
%  \label{tab:condesoptions}
%\end{table}

%Table \ref{tab:condesoptions} shows available options.

\begin{description}
\item[\texttt{w :}] In nonparametric models where the class of models are \texttt{frd}, every model has its own vector of frequency points and we do not need to assign frequency vectors in options. 
If you want to specify a frequency grid for each parametric model, define \texttt{w} as a cell such that \texttt{w\{i\}} is the vector of frequency points in which the frequency response $G_i(j\omega)$ is obtained. In other words, after you type:
\begin{lstlisting}
options = condesopt ('w' , w)
\end{lstlisting}
\texttt{options.w} will be a cell with \texttt{options.w\{1\}}, \texttt{options.w\{2\}}, \dots. Should you use just one frequency vector for all models, simply define \texttt{w} as a vector (not a cell). If you do not specify \texttt{w}, a default frequency grid for each model will be created by the \texttt{bode} command. 

\item{\tt F :}      This is a weighting filter for L-Ld. For almost all optimization problems an approximation of the two norm of F(L-Ld) is minimized. Its value is 1/(1+Ld) by default. F should be set to 1 if no filter is desired.


\item{\tt nq :}  is an integer greater than 2 representing the number of vertices of a polygon of least area that circumscribes 
the frequency domain model uncertainty circle in the Nyquist plot. If it is empty, the circle will not be approximated 
by a polygon and therefore a convex constraint is defined and an SDP solver with YALMIP is used instead of {\tt `linprog'} or {\tt `quadprog'}.    

 \item{\tt gamma :}    This option is used only for {\tt Hinf} performance, when $\gamma$ is minimized in the optimization problem (\ref{gamma_optim}).
 Minimizing the infinity norm is performed by a bisection algorithm. For multimodel case the
 maximum of $\gamma(i)$ is minimized for all models. {\tt `gamma'} is a vector containing {\tt [$g_{\min}$, $g_{\max}$, tol]} where $g_{\min}$ and $g_{\max}$ are the minimum and maximum value of gamma and {\tt tol} is a small positive number that indicates the tolerance of optimal $\gamma$.

 \item{\tt lambda :}   This option is used together with {\tt gamma} only for {\tt Hinf} performance. 
  It indicates the sum of which weighted sensitivity functions should be minimized. 
  
\item[\texttt{Gbands :}] is a string that takes two values : {\tt `on'} or {\tt `off'}. Its default value is {\tt `on'} meaning that the Gershgorin stability conditions are considered in the design of MIMO controller. If this option is turned to {\tt `off'}, the optimization problem becomes linear (for systems of more than two outputs) which had less numerical problem and is much faster than the convex solvers. However, the closed-loop stability should be verified after optimization. This option can be changed to {\tt `off'} by typing:
\begin{lstlisting}
options = condesopt ('Gbands' , 'off')
\end{lstlisting}

%\item[\texttt{beta}]  is the angle between line $d_2$ (see Fig \ref{fig:GPhC}) and the real axis. The default value of it is $45-\varphi_m/2$ degrees. This value is well-chosen and the final result of design is not very sensitive to small changes in it. (Recommendation: if you decide to change it, do it only after reading \cite{KKL07}).

\item[\texttt{np :}] This option is used only for gain scheduled controller design. {\tt np} is a vector that indicates the degree of 
polynomials describing the gain-scheduled controller parameters. For example {\tt np=[2 1]} indicates that we have two scheduling
parameters. The first one is described by a second-order polynomial and the second one is linear. Its default value is \texttt{[]} which implies that the controller is not gain-scheduled.

\item[\texttt{gs :}] This option is used if a gain-scheduled controller should be designed. \texttt{gs} is a $m$ by $n$ matrix, where $n$ is the number 
of scheduling parameters and $m$ the number of operating points. The $i$-th row of {\tt gs} contains the values of $n$ scheduling
parameters that corresponds to the $i$-th model $G_i$. The default value of \texttt{gs} is \texttt{[]}.

<<<<<<< HEAD
\item[\texttt{ntheta :}] This option is used for controllers with TF (transfer
function) structure. It is an integer greater than 2 
(default: 8) that determines the number of vertices used
to approximate the circle that determines the performance
constraints in the Nyquist diagram. If it is empty, the
the circle will not be approximated and a convex constraint
will be defined (YALMIP should therefore be used).


\item[\texttt{TFtol :}] This option is used for controllers with TF (transfer
function) structure. It should be a small, positive number.
The linear constraints are of the form Ax <= 0 and are 
replaced by Ax <= TFtol. Default: 1e-7.

=======
>>>>>>> statespace
 \item[{\tt yalmip :}]   is a string that can be set to {\tt `on'} to activate the YALMIP interface. It should be activated when  Gbands='on' for multivariable controller design of more than 2 inputs. It will be automatically activated if YALMIP has already been installed and {\tt nq} is empty. As  the default solver SDPT3 is used  but can be changed by 'solver' option, e.g. options=condesopt('yalmip','on','solver','sedumi').

\item[\texttt{Solver options :}] Besides, the options mentioned above, one can alter every option available for different solvers. In this case, one should be aware of which solver would be used in which optimization problem. For example, if one wants to design a PID controller for a SISO system without specifying a desired open loop (where optimization problem (\ref{eq:opGPhC}) is solved by \texttt{linprog}), one can change every option available for the solver \texttt{linprog} in this command. For example:
\begin{lstlisting}
options = condesopt ('w', w1, 'MaxIter', 100, 'Display', 'iter')
\end{lstlisting}
changes the values of \texttt{MaxIter} and \texttt{Display}, which are some of \texttt{linprog}'s options, to the specified values as well as assigning \texttt{w1} to \texttt{w}.

\end{description}


\subsection{Examples}

\subsubsection{$H_\infty$ controller design for an unstable system}

Consider  the family of plants described by the following multiplicative uncertainty model:
$$
	\tilde{G}(s)=\frac{(s+1)(s+10)}{(s+2)(s+4)(s-1)}[1+ W_2(s)\Delta(s)] 
$$
where 
$$
	W_2(s)=0.8\frac{1.1337s^2+6.8857s+9}{(s+1)(s+10)}
$$
The nominal performance is defined by $\| W_1 \mathcal{S} \|_\infty < 1$ with~: 
$$
	W_1(s)=\frac{2}{(20s+1)^2}
$$
Design a PID controller  to optimize  the robust performance
$$ \| |W_1 S| +  |W_2 T| \|_\infty < \gamma $$
and $\| KS \|_\infty < 20$. We choose the following $L_d(s)$ that contains the unstable pole of the plant model and meets the Nyquist stability criterion.
$$L_d(s)=2 \frac{s+1}{s(s-1)}$$
and $W_3=0.05$. Then we minimize $\gamma$ under $\| W_3 KS \|_\infty < 1$.

\begin{lstlisting}
s=tf('s');
G=(s+1)*(s+10)/((s+2)*(s+4)*(s-1));
Ld=2*(s+1)/s/(s-1);

W{1}=2/(20*s+1)^2;
W{2}=0.8*(1.1337*s^2+6.8857*s+9)/((s+1)*(s+10));
W{3}=tf(0.05);

phi=conphi('PID',0.01);
hinfper=conper('Hinf',W,Ld);
w=logspace(-1,4,500);
opt=condesopt('gamma',[0.01 2 0.001],'lambda',[1 1 0 0],'w',w);

K=condes(G,phi,hinfper,opt)

Optimization terminated.
gamma=0.78151
 
Zero/pole/gain:
16.9564 (s+2.037) (s+26.19)
---------------------------
         s (s+100)
\end{lstlisting}

The same problem can be solved without approximation of the multiplicative uncertainty circle by an octagon by setting the following options:
\begin{lstlisting}
opt.nq=[];
opt.yalmip='on';
K=condes(G,phi,hinfper,opt)

No problems detected 
gamma=0.72224
 
Zero/pole/gain:
19.0488 (s+2.227) (s+21.47)
---------------------------
         s (s+100)
\end{lstlisting}

The obtained results are better and have less conservatism but the computation time is much larger.

\subsubsection{Gain-scheduled controller design for a domestic condensing boiler}
$G_1(s)$ to $G_6(s)$ are six first order identified models concerning a domestic condensing boiler in
 different water flow rates $\theta=[8;7;6;5;4;3]$ lit./min.
 The objective is to compute a gain-scheduled PI controller with a gain margin of 2, phase margin of $60^{\circ}$.
 
 \begin{lstlisting}
s=tf('s');
G{1}= exp(-6.2*s) * 0.00932/(31.84*s + 1);
G{2}= exp(-6.02*s) * 0.01032/(34.08*s + 1);
G{3}= exp(-6.69*s) * 0.01169/(34.76*s + 1);
G{4}= exp(-9.76*s) * 0.01391/(37.62*s + 1);
G{5}= exp(-12*s) * 0.01700/(57.42*s + 1);
G{6}= exp(-15.2*s) * 0.0216/(62.17*s + 1);
              
phi=conphi('PI');
per=conper('GPhC',[2,60]);

theta=[8;7;6;5;4;3];

opt=condesopt('np',2,'gs',theta);


K=condes(G,phi,per,opt)

K{1}+theta K{2}+theta^2 K{3} 

Optimization terminated.
K{1}=
 
Zero/pole/gain:
149.8348 (s-0.01179)
--------------------
         s
 
K{2}=
 
Zero/pole/gain:
-24.775 (s-0.02971)
-------------------
         s
 
K{3}=
 
Zero/pole/gain:
5.3054 (s+0.01414)
------------------
        s
\end{lstlisting}

\subsubsection{Simultaneous stabilization (example from Robust Control Toolbox of Matlab)}

A set of seven unstable models are given:
\begin{lstlisting} 
p{1} = tf(2,[1 -2]);
p{2} = p{1}*tf(1,[.06 1]);              % extra lag
p{3} = p{1}*tf([-.02 1],[.02 1]);       % time delay
p{4} = p{1}*tf(50^2,[1 2*.1*50 50^2]);  % HF resonance
p{5} = p{1}*tf(70^2,[1 2*.2*70 70^2]);  % HF resonance
p{6} = tf(2.4,[1 -2.2]);                % pole/gain migration
p{7} = tf(1.6,[1 -1.8]); 
\end{lstlisting}
A bandwidth of 4.5 rad/s is desired leading  to the following weighting filter :

\begin{lstlisting}
desBW = 4.5; W{1}= makeweight(500,desBW,0.33);
\end{lstlisting}
A noise filter is also defined as :
\begin{lstlisting}
NF = (10*desBW)/20;  % numerator corner frequency
DF = (10*desBW)*50;  % denominator corner frequency
Wnoise = tf([1/NF^2  2*0.707/NF  1],[1/DF^2  2*0.707/DF  1]);
W{2} = Wnoise/abs(freqresp(Wnoise,10*desBW));
\end{lstlisting}
An (n+1)-th order controller with integral action is designed using a generalized basis function:

\begin{lstlisting}
phi=conphi('Generalized',logspace(-2,2,n),'s',1/s);
\end{lstlisting}
The poles of the controller are logarithmically spaced between 0.01 and 100. In the first step one initial controller is designed for $P_1(s)$: 
\begin{lstlisting}
s=zpk('s');

Ld0=10*(s+2)/s/(s-2);

n=5;
phi=conphi('Generalized',logspace(-2,2,n),'s',1/s);

per0=conper('Hinf',W,Ld0);

w=logspace(-2,3,1000);
opt=condesopt('w',w);

K0=condes(p{1},phi,per0,opt);
\end{lstlisting}
and then is used to compute $L_d(s)$ for all models:
\begin{lstlisting}
for j=1:7,    
    Ld1{j}=K0*p{j};
    per{j}=conper('Hinf',W,Ld1{j});
end
\end{lstlisting}
Then an $H_\infty$ controller can be designed using the following commands:
\begin{lstlisting}
opt.gamma=[0.1,5,0.01];
K=condes(p,phi,per,opt);  
\end{lstlisting}
$$K= \frac{385.9471 (s+12.46) (s+1.679) (s+0.8785) (s+0.14) (s+0.009792)}{s (s+0.01) (s+0.1) (s+1) (s+10) (s+100) }$$
               

It is interesting to see that the same code can be used for computing a second order controller ($n=1$) that gives also satisfactory results (robust control toolbox of Matlab ends up with a sixth order controller). For $n=1$ we obtain:
$$
K=\frac{325.3633 (s+2.154)}{s (s+100)}
$$
The disturbance response of the closed-loop system is given for the second-order controller and 6-th order controller for the sake of comparison with the results of robust control toolbox of Matlab.
\begin{center}
\includegraphics[width=10cm]{2ndorder.eps} \\
Disturbance response of the 2nd order controller\\

\includegraphics[width=10cm]{6thorder.eps} \\
Disturbance response of the 6-th order controller
\end{center}



%\chapter{Robust Controller Design Examples}
%
%\section{Example 1}
%
%Design a PID controller for the following first order model with delay:
%$$G=\frac{e^{-s}}{(s+1)^3}$$
%The objective is to ensure: Gain margin = 2, Phase margin = $60^{\circ}$, Crossover frequency = 0.08 rad/s. 
%
%\begin{lstlisting}
%s=tf('s');
%G=exp(-s)/(s+1)^3;
%
%phi=conphi('PID');
%per=conper('GPhC',[2,60,.08]);
%
%K=condes(G,phi,per)
% 
% K =
%  
%   174 s^2 - 16.5 s + 7.769
%   ------------------------
%        s^2 + 83.33 s
%  
%\end{lstlisting}
%
%\section{Example 2}
%Design a PID controller for the following first order model with delay:
%$$G=\frac{e^{-5s}}{s(s+1)^3}$$
%for the following specifications:
% Gain margin = 2, Phase margin = $60^{\circ}$, Crossover frequency = 0.09 rad/s, Desired open loop: $L_d(s)=0.1/s^2$. 
%
%\begin{lstlisting}
%s=tf('s');
%G{1}=exp(-5*s)/(s*(s+1)^3);
%phi=conphi('PID');
%per=conper('GPhC',[2,60,.09],.1/s^2);
%
%K=condes(G,phi,per)
%
%K=
%
%46 s^2 + 7.158 s + 0.2987
%-------------------------
%         s^2 + 83.33 s
%\end{lstlisting}
%
%\section{Example 3}
%Consider a system that has three different models in three operating points as follows:
%$$G_1=\frac{4e^{-3s}}{(10s+1)} \qquad G_2=\frac{e^{-5s}}{(s^2+14s+7.5)}  \qquad G_3=\frac{2e^{-s}}{(20s+1)}$$
%The objective is to design a robust PID controller to ensure a gain margin of 3, phase margin of 60 for all models with different desired crossover frequencies as follows: $\omega_{c_1}$=0.19 rad/s (for model 1), $\omega_{c_2}$=0.007 rad/s (for model 2), $\omega_{c_3}$=0.071 rad/s (for model 3).
%
%We consider $N=400$ logarithmically spaced points in interval: $[0.001,20]$ and $\tau=0.05$.
%\begin{lstlisting}
%s=tf('s');
%G{1}=exp(-3*s)*4/(10*s+1);
%G{2}=exp(-5*s)/(s^2+14*s+7.5);
%G{3}=exp(-s)*2/(20*s+1);
%
%freq = logspace(log10(.001),log10(20),400);
%
%phi=conphi('PID',.05);
%
%per{1}=conper('GPhC',[3,60,.19]);
%per{2}=conper('GPhC',[3,60,.007]);
%per{3}=conper('GPhC',[3,60,.071]);
%
%options = condesopt ('w',freq);
%K=condes(G,phi,per,options)
%
% K =
%  
%   11.54 s^2 + 10.35 s + 1.077
%   ---------------------------
%           s^2 + 20 s
%\end{lstlisting}
%
%\section{Example 4}
%Consider two MIMO models given by:
%$$ G_1(s)=\left[ \begin{array}{cc}
%\frac{5e^{-3s}}{4s+1}  &  \frac{2.5e^{-5s}}{15s+1} \\[.25 cm]
%\frac{-4e^{-6s}}{20s+1}  &  \frac{e^{-4s}}{5s+1} 
%\end{array} \right] \qquad G_2(s)=\left[ \begin{array}{cc}
%\frac{10e^{-6s}}{8s+1}  &  \frac{5e^{-10s}}{30s+1} \\[.25 cm]
%\frac{-8e^{-12s}}{40s+1}  &  \frac{2e^{-8s}}{10s+1} 
%\end{array} \right]$$
%The objective is to design a gain scheduled MIMO PI controller to achieve a modulus margin of 0.5 for the diagonal open-loop systems and a desired open-loop transfer function:
%$$L_d(s)=\left[ \begin{array}{cc}
%\frac{1}{30 s}  &  0 \\[.25 cm]
%0  &  \frac{1}{30 s} 
%\end{array} \right] $$
%We suppose that the first model corresponds to a scheduling parameter $\theta_1=-1$ and the second model to $\theta_2=1$. 
%\begin{lstlisting}
%s=tf('s');
%G{1}=[5*exp(-3*s)/(4*s+1) 2.5*exp(-5*s)/(15*s+1); -4*exp(-6*s)/(20*s+1) exp(-4*s)/(5*s+1)];
%G{2}=[10*exp(-6*s)/(8*s+1)  5*exp(-10*s)/(30*s+1); -8*exp(-12*s)/(40*s+1) 2*exp(-8*s)/(10*s+1)];
%
%phi=conphi('PI'); 
%
%per=conper('LS',0.5,1/(30*s)); 
%
%options = condesopt ('Gbands','off', 'np',1,'gs',[-1;1]);
%
%K=condes(G,phi,per,options)
%
% K{1}+theta_1 K{2}
% 
% K{1}=
%   
%   From input 1 to output...
%        0.0561 s + 0.001666
%    1:  -------------------
%                 s
%  
%        0.08391 s + 0.00667
%    2:  -------------------
%                 s
%  
%   From input 2 to output...
%        -0.06813 s - 0.00417
%    1:  --------------------
%                 s
%  
%        0.2929 s + 0.008327
%    2:  -------------------
%                 s
%  
% 
% K{2}=
%  
%   From input 1 to output...
%        -0.0001459 s - 0.0005556
%    1:  ------------------------
%                   s
%  
%        -0.0004204 s - 0.002222
%    2:  -----------------------
%                   s
%  
%   From input 2 to output...
%        0.0009708 s + 0.001388
%    1:  ----------------------
%                  s
%  
%        -0.001374 s - 0.002778
%    2:  ----------------------
%                  s
%  
% 
%\end{lstlisting}
%
%\section{Example 5}
%Consider  the family of plants described by the following multiplicative uncertainty model:
%$$
%	\tilde{G}(s)=\frac{(s+1)(s+10)}{(s+2)(s+4)(s-1)}[1+ W_2(s)\Delta(s)] 
%$$
%where 
%$$
%	W_2(s)=0.8\frac{1.1337s^2+6.8857s+9}{(s+1)(s+10)}
%$$
%The nominal performance is defined by $\| W_1 \mathcal{S} \|_\infty < 1$ with~: 
%$$
%	W_1(s)=\frac{2}{(20s+1)^2}
%$$
%Design a PID controller  to optimize  the robust performance
%$$ \| |W_1 S| +  |W_2 T| \|_\infty < \gamma $$
%and $\| KS \|_\infty < 20$. We choose the following $L_d(s)$ that contains the unstable pole of the plant model and meets the Nyquist stability criterion.
%$$L_d(s)=2 \frac{s+1}{s(s-1)}$$
%and $W_3=0.05$. Then we minimize $\gamma$ under $\| W_3 KS \|_\infty < 1$.
%
%\begin{lstlisting}
%s=tf('s');
%G=(s+1)*(s+10)/((s+2)*(s+4)*(s-1));
%Ld=2*(s+1)/s/(s-1);
%
%W{1}=2/(20*s+1)^2;
%W{2}=0.8*(1.1337*s^2+6.8857*s+9)/((s+1)*(s+10));
%W{3}=tf(0.05);
%
%phi=conphi('PID',0.01);
%hinfper=conper('Hinf',W,Ld);
%opt=condesopt('gamma',[0.01 2 0.001],'lambda',[1 1 0 0])
%
%K=condes(G,phi,hinfper,opt)
%
%Optimization terminated.
%gamma=1.006
% 
%Transfer function:
%16.58 s^2 + 138 s + 204.2
%-------------------------
%       s^2 + 100 s
%\end{lstlisting}

\bibliographystyle{plain}
\bibliography{/Users/akarimi/Documents/Karimi/papers/bibfiles/linear}
<<<<<<< HEAD
\end{document}


%%% Local Variables:
%%% mode: latex
%%% TeX-master: t
%%% End:
=======
\end{document}
>>>>>>> statespace
